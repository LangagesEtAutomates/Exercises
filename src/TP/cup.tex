\documentclass{td}

\inputexercisepath[src/corrections]{src/exercises}

\usepackage[hidelinks]{hyperref}
\usepackage{amsmath,amssymb, mathrsfs, stmaryrd}
\newcommand\eqdef{\stackrel{\text{def}}{=}}

\usepackage{statemachines}
\usepackage{trees}
\usetikzlibrary{decorations.pathmorphing}

\usepackage{src/sty/config}

\codeUE{XLG4IU020}
\intituleUE{Langages et Automates}
\cursus{L2 Informatique \& Info-maths}

\author{Matthieu \textsc{Perrin}}

\logo{src/img/logoUN.png}
\institution{Nantes Université}

\typeTP[2]
\title{Utilisation de JFlex}

\hypersetup{
  pdftitle  = {Langages et Automates - TP 2 - Flex},
  pdfauthor = {Matthieu Perrin},
  pdfsubject  = {TP de L2 sur l'analyse lexicale avec JFlex},
  pdfkeywords = {automates finis, expressions régulières, analyse lexicale}
}

\begin{document}

\maketitle

\textbf{Flex} (ou \textbf{lex} sur Unix) est un générateur d'analyseurs lexicaux.
Il prend un fichier en argument contenant un \textbf{système de définitions rationnelles} et
une \textbf{liste d'expressions rationnelles}
et retourne un fichier contenant un programme en langage C (principalement
une fonction se nommant \textbf{yylex})
pouvant découper un flux de caractères en une suite de
\textbf{tokens} (des sous-chaînes de caractères formant des unités lexicales)
suivant ces expressions rationnelles.
Si, pour la position courante du flux de caractères en entrée, plusieurs tokens de longueurs
différentes sont possibles, le token de \textbf{longueur maximale} sera sélectionné.

Chaque expression rationnelle est associée à
une action qui sera exécutée pour chaque token qui
lui correspond (si le token correspond
a plusieurs expressions rationnelles, seule l'action de la première
expression rationnelle
dans la liste des expressions rationnelles sera exécutée).
Si l'expression rationnelle n'est pas associée explicitement
à une action dans le fichier source de \textbf{flex}, l'action par défaut consiste à afficher
le token.
Une action peut-être
une instruction ou
un bloc d'instructions (délimité par \textbf{\{} et \textbf{\}}) ou bien une action prédéfinie
(un mot-clé en majuscule).

\textbf{Flex} est particuliérement bien adapté pour son utilisation conjointe
avec le générateur d'analyseur syntaxique \textbf{bison}.
Dans ce cas, le flux
d'entrée est d'abors décomposé en un flux de tokens
par l'analyseur lexical construit par \textbf{flex} puis ce flux passe par
l'analyseur syntaxique généré par \textbf{bison}. De nombreux programmes
utilisent des programmes générés par \textbf{flex} et/ou \textbf{bison} comme
des compilateurs (cc, gcc), des interpréteurs de commandes (sh, csh),
des calculateurs de bureau (bc)...

Les versions d'origine de ses générateurs ont été écrites dans les années
70 dans les laboratoires Bell de AT\&T par Stephen C. Johnson (yacc),
Mike Lesk et Eric Schmidt (lex). Dans les TP, nous utiliserons 
\textbf{flex} et \textbf{bison},
les implémentations de la Free Software Foundation (le groupe GNU).

\subsection*{Premier programme}

\textbf{Flex} est un générateur de programme pouvant générer du code C
ou C++. Ici, nous utiliserons le C. Pour l'utiliser, nous
devons taper (avec notre éditeur de texte
préféré) le fichier source contenant les expressions rationnelles des
tokens et les actions associées. Voici un petit exemple
que vous pouvez sauvegarder sous le nom \textbf{ex.l} (on utilise
l'extenxion \textbf{.l} pour désigner les fichiers sources de \textbf{flex})~:
\begin{lstlisting}
  %%
  if+    printf("si");
  else   printf("sinon");
  %%
  int main() {
    yylex();
  }
\end{lstlisting}

Ce fichier est ensuite traité par le programme \textbf{flex}
pour qu'il le transforme en un programme C. On peut utiliser la
commande suivante pour cela (à taper dans un terminal)~:
\begin{lstlisting}
  flex ex.l
\end{lstlisting}

Si tout va bien (c'est-à-dire si \textbf{flex} n'a pas détecté d'erreur),
vous devez trouver le fichier \textbf{lex.yy.c} dans le répertoire
courant. On peut alors compiler ce programme
en utilisant un compilateur C (sans oublier de donner l'option
\textbf{-lfl} pour ajouter la librairie \textbf{flex}):
\begin{lstlisting}
  gcc lex.yy.c -lfl -o ex
\end{lstlisting}

Le résultat (le fichier exécutable) porte le nom \textbf{ex}
(cela correspond à l'option 
\textbf{-o ex}).
Cet exécutable lit les
caractères sur son entrée standard et écrit le résulat des actions sur sa
sortie standard. La commande suivante lance ce programme avec pour
entrée standard la chaîne construite par la commande \textbf{echo}
(le résultat s'affiche sur le terminal)~:
\begin{lstlisting}
  echo "ifff Hello then Ok else Bye" | ./ex
\end{lstlisting}
On peut aussi mettre le texte à traiter dans un fichier, par exemple
le fichier \textbf{fichier.txt}, en utilisant un éditeur de texte,
et utiliser l'opérateur \textbf{<} de redirection de l'entrée standard~:
\begin{lstlisting}
  ./ex < fichier.txt
\end{lstlisting}

Normalement, vous devez voir, sur votre écran la
chaîne de la commande \textbf{echo} ou le contenu du fichier
\textbf{fichier.txt} dans lesquels les occurences de ``if'',
``iff'', ``ifff'', ``iffff'', etc ont
été remplacées par ``si'' (l'expression rationnelle
\textbf{if+} correspond à un ``i'' suivi de un ou plusieurs ``f'')
et celles de ``else'' par ``sinon''.

En fait, le fichier \textbf{ex.l} est transformé par \textbf{flex} en un programme C contenant
la fonction \textbf{main} définie dans la dernière partie de \textbf{ex.l}. Cette fonction
principale appelle la fonction \textbf{yylex} qui a été générée
par \textbf{flex} à partir de la spécification suivante~:
\begin{lstlisting}
  if+    printf("si");
  else   printf("sinon");
\end{lstlisting}

Cette definition précise que si l'analyseur reconnaît un token
correspondant à \textbf{if+} (un ``i'' suivi de un ou plusieurs
``f''), il doit exécuter l'action \textbf{printf("si");} puis
continuer l'analyse au caractère suivant. S'il
reconnaît le token \textbf{else}, il doit exécuter \textbf{printf("sinon");}
et continuer l'analyse u caractère suivant. Sinon, il doit afficher le prochain caractère
et continuer l'analyse au caractère suivant (token par défaut avec l'action par défaut).
La fonction s'arrête lorsqu'on arrive à la fin du fichier
d'entrée.

\subsection*{Définition des fichiers sources de \textbf{flex}}

\noindent Un fichier source \textbf{flex} est composé de trois parties
délimitées par les caractères \lstinline:

  \noindent
  \hspace*{2cm}Définition\\
  \hspace*{2cm}\lstinline\\
    \hspace*{2cm}Règles\\
    \hspace*{2cm}\lstinline\\
      \hspace*{2cm}Fonctions et routines

      Toutes les lignes précédant le premier délimiteur 
      \lstinline sont considérées comme une définition
        par \textbf{flex} ou bien, pour les lignes comprises entre
        \verb|%{| et \verb|%}| ou celles
        débutant par un espace ou une tabulation,
        comme du code C copié
        tel quel dans le fichier résultat. Une définition
        sert à introduire des abréviations pour des motifs
        pouvant apparaître dans les règles. Ainsi, on
        peut spécifier l'ensemble des chiffres par
        la définition \verb|D [0-9]| et celle
        d'exposant pour un nombre par la définition
        \verb|E [DEde][-+]?{D}+|. Ces définitions
        peuvent alors être utilisées dans une règle 
        comme une sorte de macro en utilisant {\it \{nom\}}
        avec {\it nom} le nom de la définition (ci-dessus
        \verb|{D}| et \verb|{E}| respectivement pour un
        chiffre et un exposant).

        La seconde partie d'un source \textbf{flex}, délimitée avant et après
        par \verb|%%| comporte une liste de règles lexicales
        dont chacune est composée de deux parties:
        \begin{itemize}
        \item un {\bf motif} ({\it pattern} en anglais) représentant
          une expression régulière. Une suite de caractères donnée correspond
          ou non à un motif. Par exemple, le motif \verb|[A-Za-Z]+|
          correspond à toutes les suites d'au moins un caractère et
          composées seulement de lettres. Les motifs doivent débuter une
          ligne (pas d'espace ni de tabulation).
        \item une {\bf action} associée au motif et qui est exécutée
          lorsque l'analyseur reconnaît le motif. Cett action peut être
          une instruction C simple comme \verb|printf("si");|, un  bloc
          d'instruction C délimité par \verb|{| et \verb|}| (sans \verb|;|
          à la fin), une action prédéfinies comme \verb|ECHO;| ou bien
          être vide en donnant un simple point-virgule \verb|;|. Si
          l'action ne comporte pas d'instruction {\tt return},
          d'appel récursif à l'analyseur {\tt yylex();} et
          si le buffer d'entrée n'est pas modifié, l'analyseur
          continue avec la lecture du prochain token. On peut associer
          la même action à plusieurs motifs en utilisant le caractère
          \verb=|= à la place de l'action.
        \end{itemize}

        Finalement, la dernière partie d'un source \textbf{flex}, après le second
        \verb|%%|, est recopié tel quel à la fin du programme
        \texttt{lex.yy.c}.
        Il sert à définir des fonctions
        utiles à l'analyseur. Dans le premier programme, nous
        y avions défini la fonction \verb|main| pour obtenir un
        programme C complet.
        Voici un exemple de source \textbf{flex}:
        \begin{lstlisting}
          %{
          int nombres;
          int mots;
          %}
          D                  [0-9]
          E                  [Ee][-+]?{D}+
          %%
          {D}+\.?{D}*{E}?    |
          \.{D}+{E}?         nombres++;
            [^ \n]*@[^ \n]*        ECHO;
            [A-Za-z]+          { mots++; }
            .|[\n]             ;
            %%
            int main() {
              nombres = 0;
              mots = 0;
              yylex();
              printf("\nNombres : %d, Mots : %d\n", nombres, mots);
            }
        \end{lstlisting}

        Ce fichier source \textbf{flex} déclare deux variables globales
        {\tt nombres} et {\tt mots}. Ensuite, on trouve les
        deux définitions pour \verb|{D}| et \verb|{E}|.
        Après, nous avons les différentes règles pour les tokens. Les deux
        premiers motifs \verb|{D}+\.{D}*{E}?| et 
        \verb|\.{D}+{E}?| reconnaissent des nombres (avec
        exposant optionnel). Ils déclenchent la même action qui
        consiste à incrémenter la variable \verb|nombres| de
        un. Le troisième motif \verb|[^ \n]*@[^ \n]*| utilise l'action
        prédéfinies \verb|ECHO| qui affiche le token sur l'écran
        (quelque chose qui ressemble à une adresse électronique).
        Le motif suivant reconnaît un mot (enfin une suite de lettres
        non-accentuées -- les lettres accentuées divise le mot
        en plusieurs parties). On incrémente 
        \verb|mots| de un. Enfin, le denier motif est constitué
        de n'importe quel caractère. Dans ce cas, aucune action n'est réalisée.
        Cela permet de supprimer l'action par défaut
        qui affiche le caractère courant lorsqu'aucun
        motif ne convient.

        \begin{exercice}[Premiers pas avec \texttt{JFlex}]

          \begin{question}
          \item Cloner le dépôt~:\\
            \texttt{git clone https://github.com/LangagesEtAutomates/TP2-Template}
          \item Générer les analyseurs et compiler~:\\
            \texttt{cd TP2-Template}\\
            \texttt{ant compile}
          \item Lancer l'exo 1~:\\
            \texttt{java -cp "build:lib/*" exo1.Main}
          \end{question}


          
          \begin{question}
          \item Ouvrir le fichier \texttt{TP2-Template/src/exo1/lexer.flex} et le lire. 
            

            
          \item Ajouter une règle pour \texttt{alors} affichant \verb|[ALORS]|.
          \item Ajouter des règles pour~:
            \begin{itemize}
            \item les nombres (\verb|[0-9]+(\.[0-9]+)?([eE][+-]?[0-9]+)?|)~$\rightarrow$~\verb|[NUM:...]|;
            \item les identifiants (\verb|[A-Za-z_][A-Za-z_0-9]*|)~$\rightarrow$~\verb|[ID:...]|;
            \item les espaces (\verb|[ \t\r\n]+|)~$\rightarrow$~recopie du texte.
            \end{itemize}
          \item Ordonner les règles ainsi~:\\
            \texttt{si}, \texttt{sinon}, \texttt{alors}, \texttt{\{NUM\}}, \texttt{\{ID\}}, \texttt{\{WS\}}, \texttt{.}, \texttt{\texttt{<}\texttt{<}EOF\texttt{>}\texttt{>}}.
          \item (\emph{Optionnel}) Ajouter \verb|%char| et afficher la position du jeton avec \verb|yychar|.

          \end{question}

          \noindent\textbf{Test~:}
\begin{verbatim}
si sinon alors iff if1 _x 123 12.34 1e-3 bonjour++
\end{verbatim}

\noindent\textbf{Attendu~:}
\verb|[SI] [SINON] [ALORS] [ID:iff] [ID:if1] [ID:_x] [NUM:123] [NUM:12.34] [NUM:1e-3] [ID:bonjour]++|

        \end{exercice}



        \begin{exercice}
          Saisir ce second source \textbf{flex} puis le tester en utilisant un fichier d'entrée
          contenant des nombres avec ou sans exposant (comme \texttt{234.2}, \texttt{54},
          \texttt{202.8e10} ou \texttt{1E-34}), des adresses
          électoniques (comme toto.durant@etu.univ-nantes.fr) et des mots sans accent 
          (\textbf{flex} n'utilise pas le codage UTF-8 -- Pour lui, un caractères
          est un octet du flux d'entrée -- Il ne regroupe pas les octets 
          par caractère Unicode -- On peut utiliser des expressions rationnelles
          comprenant des caractères accentués mais ils ne doivent pas être utilisés
          dans des classes de caractères).
        \end{exercice}
        \begin{correction}
          Rien à corriger.
        \end{correction}


        \subsection*{Motifs}

        Les motifs sont des expressions régulières 
        comme pour la commande
        \texttt{grep} (voir le TP 1).
        Quelques spécificités sont toutefois à noter~:
        \begin{itemize}
        \item Une suite de caractères correspond à la même suite de
          caractères dans le flux d'entrée. On peut placer ces caractères
          (ou une partie seulement)
          entre  guillemets (en particulier, cela est utile si l'on
          inclut un blanc, une tabulation ou un opérateur \textbf{flex}
          \verb=[ ] ^ - ? . * + | [ ] $ / { } % < >=. Ces caractères
          perdent leur qualité d'opérateur lorsqu'ils sont précédé 
          de \verb|\|. Le caractère \verb|\| s'obtient en écrivant \verb|\\|.
          Une tabulation se code \verb|\t|, le passage
          à la ligne \verb|\n| et \verb|"| par  \verb|\"|.
          Le caractère \verb|\| s'obtient en écrivant \verb|\\|.
          Par exemple, le motif
          \verb|xyz"+\"+"\n| correspond aux chaînes constituées
          des caractères \verb|x y z + " +| et finissant par le caractère
          ``passage à la ligne'' (cette chaîne doit donc se trouver
          en fin de ligne).
        \item Une classe de caractères est indiquée avec des crochets.
          Ainsi \verb|[+a-z0-9]| correspond au caractère \verb|+|, 
          à toute lettre minuscule ou à tout chiffre. 
          Dans \verb|a-z| et \verb|0-9|, le symbole \verb|-|
          a permis de spécifier tous les caractères compris entre deux
          caractères . Si l'on désire ce 
          caractère \verb|-|, il faut le placer en tête de la liste comme
          dans \verb|[-+0-9]|.
          %On peut placer
          %un blanc entre crochet sans le faire précéder de \verb|\|.
          On obtient la liste opposée des caractères spécifiés en ajoutant
          \verb|^| après le \verb|[|. Ainsi, \verb|[^abc]| correpond à tous
          les caractères sauf \verb|a|, \verb|b| et \verb|c|, même
          les caractères de contrôle comme les passages à la ligne.
        \item \verb|.| correspond à un caractère
          sauf le passage à la ligne. Il est donc équivalent à l'expression
          \verb|[^\n]|.
        \item \verb|?| zéro ou une occurrence
          de l'expression qui le précède.
        \item \verb|*| spécifie la répétition
          d'un nombre quelconque de fois d'une expression.
        \item \verb|+| spécifie une répétition
          {\em non vide} d'une expression.
        \item \verb=|= permet d'avoir une alternance
          entre plusieurs motifs.
        \item \verb|(| et \verb|)|
          permettent de grouper des expressions
          comme dans \verb=(ab|cd+)?(ef)*= qui correspond
          à des chaînes comme \verb|abefef|, \verb|efefef|,
          \verb|cdef| ou \verb|cddd| mais pas à
          \verb|abc|, \verb|abcd| ou \verb|abcdef|.
        \item \verb|^| au début du motif indique
          que le motif ne peut être reconnu qu'en début de ligne.
        \item \verb|$| à la fin du motif
          spécifie une reconnaissance en fin de ligne (juste
          avant un caractère ``passage à la ligne'').
        \item \verb|/| permet de spécifier un contexte
          à droite. Ainsi \verb|ab/cd| reconnaît les caractères
          \verb|ab| seulement s'ils sont suivis des caractères \verb|cd|.
          En fait, le motif \verb|ab$| est équivalent à \verb|ab/\n|.
        \item \verb|{|{\it nom}\verb|}| utilise la définition de {\it nom}.
        \item \verb|{|$n$\verb|}| (\verb|{|$n$\verb|,|$m$\verb|}|)
          correspond à un motif répété
          $n$ fois (de $n$ à $m$ fois). Ainsi \verb|a{1,5}| correpond à une
          chaîne composée de 1 à 5 \verb|a|.
        \end{itemize}


        \begin{exercice}
          Ecrire et tester un source \textbf{flex} permettant
          de reconnaître les motifs suivants. Ce programme devra afficher
          les motifs reconnus et seulement ceux-ci ainsi que le numéro
          du motif reconnu entre parenthèses.
          \begin{enumerate}
          \item Le mot \verb|Yacc|.
          \item Le mot \verb|From| en début de ligne.
          \item Une ligne composée d'un signe \verb|+| répété entre 3 et 10 fois.
          \item Un identificateur composé de majuscules, minuscules, du
            caractère \verb|_| et de chiffres mais ne commençant
            pas par un chiffre.
          \item un nombre avec virgule flottante.
          \end{enumerate}
        \end{exercice}
        \begin{correction}
          \begin{lstlisting}
            D [0-9]
            E [Ee][-+]?{D}+
            %%
            Yacc               ECHO; { printf("(1)\n");}
            ^From              ECHO; { printf("(2)\n");}
            ^[+]{3,10}$        ECHO; { printf("(3)\n");}
            [A-Za-z_][A-Za-z_0-9]*  ECHO; { printf("(4)\n");}
            {D}+\.?{D}*{E}?    |
            \.{D}+{E}?         ECHO; { printf("(5)\n");}
            .|\n               ;
            %%
            int main() {
              yylex();
            }
          \end{lstlisting}
        \end{correction}






        \subsection*{Reconnaissance}

        La reconnaissance par \textbf{flex} s'arrête lorsqu'il arrive à la
        fin du flux. La fonction \verb|yylex| retourne alors la valeur $0$.
        On peut aussi sortir de cette fonction par une instruction
        \verb|return| dans une action.

        Lorsqu'on spécifie une suite de motif, il se
        peut que plusieurs motifs différents peuvent
        s'appliquer au même moment et cela
        sur des chaîne de longueurs différentes.
        Par exemple, avec les deux motifs \verb|a[bc]+|
        et \verb|ac*| et la chaîne \verb|accbcade|
        comme flux d'entrée, les chaînes \verb|ac|,
        \verb|acc|, \verb|accb| et \verb|accbc|
        correspondent au premier motif tandis que
        \verb|a|, \verb|ac| et \verb|acc|
        correspondent au second. \textbf{Flex choisit toujours
          le motif pour avoir la chaîne reconnue la
          plus longue}. Dans l'exemple, cela serait
        le pemier motif et le token serait
        \verb|accbc|. \textbf{Si deux motifs permettent
          d'obtenir le token le plus long, alors c'est l'action
          de la première règle qui est choisie}. L'ordre
        des motifs dans le source \textbf{flex}
        a donc une importance dans ce cas (mais
        pas si le choix des motifs donne des tokens
        de longueurs différentes).


        \subsection*{Actions}

        Nous avons vu qu'une action peut être une instruction ou
        un bloc C ou bien une action prédéfinie. Dans une action,
        on peut utiliser les variables suivantes:
        \begin{itemize}
        \item \verb|yytext| de type \verb|char*| et pointant sur
          le token reconnu (le premier caractère).
        \item \verb|yyleng| de type \verb|int| et retournant
          la longueur du token.
        \end{itemize}

        On peut replacer une part des caractères composant le
        token en utilisant la fonction \verb|yyless(nb)| qui
        replace \verb|nb| caractères dans le flux d'entrée.
        Par exemple, \verb|yyless(yyleng-1)| ne conserve
        que le premier caractère du token. L'analyse continuera
        donc avec le second caractère.

        La routine \verb|yymore();| permet d'indiquer que
        l'on doit conserver le token reconnu dans le buffer
        et le concaténer avec le token a reconnaître.

        On peut utiliser l'instruction \verb|return| $n$\verb|;| dans
        une action ce qui retourne la valeur de $n$ à la
        fonction qui a appelé \verb|yylex|. Un appel postérieur
        à cette fonction continuera l'analyse lexicale avec
        le token suivant.

        \begin{exercice}
          Ecrire un programme qui compte le nombre d'octets, de mots
          et de lignes d'un fichier (comme le programme \verb|wc|).
        \end{exercice}
        \begin{correction}
          \begin{lstlisting}
            %{
            int octets, mots, lignes;
            %}
            %%
            [A-Za-z]+  { mots++; octets+=yyleng; }
            \n         { lignes++; octets++; }
            .          { octets++; }
            %%
            int main() {
              octets = 0;
              mots = 0;
              lignes = 0;
              yylex();
              printf("\nOctets : %d, Mots : %d, Ligne : %d\n", octets, mots, lignes);
            }
          \end{lstlisting}
        \end{correction}


        \begin{exercice}

          Ecrire un programme qui transforme les nombres entiers en base
          16 (format \verb|0[xX][0-9a-fA-F]+|) par un entier en décimal. 
          Par exemple, \verb|0x10| deviendra \verb|16|, \verb|0x1ff|
          devriendra \verb|511|. Vous pouvez utiliser la fonction
          \textbf{strtol} pour faire la conversion.

        \end{exercice}

        \begin{correction}
          \begin{lstlisting}
            %%
            \verb|0[xX][0-9a-fA-F]+  { printf("%d", strtol(yytext+2, 0, 16)); }
              %%
              int main() {
                yylex();
              }
          \end{lstlisting}
        \end{correction}


\end{document}
