% SPDX-License-Identifier: CC-BY-SA-4.0
% Part: Syntactic analysis
% Section: Decision algorithms
% Exercise: Strings recognized by pushdown automata

\begingroup

\begin{exercice}[Chaînes reconnues par un automate à pile]\label{exo:parsing/decision/pushdown_definitions}

  Soit $A$ l'automate à pile : 
  \begin{tikzpicture}[pushdown, baseline=(0.base)]
    \state[initial, accepting] (0) at (0,0) {$0$}; 
    \state                     (1) at (1,0) {$1$};
    
    \path (0) edge[loop above] node {\smPAtrans{a}{\varepsilon}{A}}                     (0);
    \path (1) edge[loop above] node {\smPAtrans{b}{A}{\varepsilon}}                     (1);
    \path (0) edge[bend left]  node {\smPAtrans{\varepsilon}{\varepsilon}{\varepsilon}} (1);
    \path (1) edge[bend left]  node {\smPAtrans{\varepsilon}{\diamond}{\varepsilon}}    (0);
  \end{tikzpicture}

  On rappelle les trois critères d'acceptation de $A$ : 
  \begin{itemize}
  \item $\mathcal{L}_F(A)$ est le langage reconnu par état accepteur.
  \item $\mathcal{L}_\varepsilon(A)$ est le langage reconnu par pile vide.
  \item $\mathcal{L}(A)$ est le langage reconnu par état accepteur et pile vide.
  \end{itemize}

  Donner le graphe de la relation d'action entre toutes les configurations accessibles
  dans la reconnaissance de chacun des mots suivants et en déduire s'ils sont acceptés
  par état accepteur, par pile vide et par état accepteur et pile vide.
  \begin{question}
  \item $ab$
  \item $aba$
  \item $abab$
  \item Décrire les langages $\mathcal{L}_F(A)$, $\mathcal{L}_\varepsilon(A)$ et $\mathcal{L}(A)$.
  \end{question}

\end{exercice}

\endgroup
\endinput
