% SPDX-License-Identifier: CC-BY-SA-4.0
% Part: Syntactic analysis
% Section: Algebraic grammars
% Challenge exercise: Dangling else

\begingroup

\begin{exercice}[Mini-défi -- Ambiguïté du \og dangling else \fg]\label{exo:parsing/grammars/challenge}

  On considère une grammaire $G$ simplifiée des instructions conditionnelles du C :

  $\begin{array}{rrl}
    \mathit{S} &\rightarrow& \mathit{Instruction} \\
    \mathit{Instruction} & \rightarrow & \text{ ``if(''} \mathit{Condition} \text{ ``)'' } \mathit{Instruction}\\
    &\mid& \text{ ``if(''} \mathit{Condition} \text{ ``)'' } \mathit{Instruction} \text{ ``else'' } \mathit{Instruction} \\
    &\mid& \text{ ``;''} \\
    \mathit{Condition} &\rightarrow& \text{``true''} \mid \text{``false''} \\
  \end{array}$

  \begin{question}
  \item Montrez, en donnant deux arbres de dérivation distincts pour le même mot engendré, que la grammaire $G$ est ambiguë. 
  \item Modifiez la grammaire pour lever l’ambiguïté. 
  \item Justifiez que la nouvelle grammaire n’est pas ambiguë.
  \end{question}

\end{exercice}

\endgroup
\endinput
