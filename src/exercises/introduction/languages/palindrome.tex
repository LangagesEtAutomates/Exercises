% SPDX-License-Identifier: CC-BY-SA-4.0
% Part: Introduction
% Section: Words and languages
% Exercise: Palindromes and Fibonacci sequence

\begingroup

\begin{exercice}[Palindrome]\label{exo:introduction/languages/palindrome}
  Un palindrome sur un alphabet $\Sigma$ est un mot pouvant être lu indifféremment de gauche à droite ou de droite à gauche (par exemple : Laval, été).
  Formellement, un mot $u \in \Sigma^\star$ est un palindrome si
  $$\forall i \in \llbracket |u| \rrbracket, u[i] = u[1+|u| - i].$$

  On définit sur $\Sigma =\{a,b\}$ la suite de mots $(f_n)_{n>0}$ (suite de Fibonacci) de la façon suivante :
  $$
  \left\{\begin{array}{rcll}
  f_1 &=& a\\
  f_2 &=& ab \\
  f_{n+2} &=& f_{n+1}\cdot f_n & \text{si $n>0$}
  \end{array}
  \right.
  $$
  
  \begin{question}
  \item Montrer que pour tout $i \ge 2$, il existe un palindrome $u_i$ tel que:
    $$ f_i = \left \{\begin{array}{ll}
    u_i\cdot a\cdot b & \mbox{si $i$ pair} \\
    u_i\cdot b\cdot a & \mbox{si $i$ impair}
  \end{array}\right.$$
  \end{question}

\end{exercice}

\endgroup
\endinput
