% SPDX-License-Identifier: CC-BY-SA-4.0
% Part: Introduction
% Section: Words and languages
% Challenge exercise: Levi Lemma

\begingroup

\begin{exercice}[Mini-défi -- Lemme de Levi]\label{exo:introduction/languages/challenge}

  Soit $\Sigma$ un alphabet. 
  \begin{question}
  \item Soient $t,u,v,w \in \Sigma^\star$ tels que $t \cdot u = v \cdot w$. 
    Montrer qu'il existe un unique $z \in \Sigma^\star$ tel que :
    \begin{itemize}
    \item soit $u = z \cdot w$ et $v = t \cdot z$,
    \item soit $t = v \cdot z$ et $w = z \cdot u$.
    \end{itemize}
  \item En déduire que pour tous $u,v,w \in \Sigma^\star$,  
    si $u \in \mathrm{Pref}(w)$ et $v \in \mathrm{Pref}(w)$ alors  
    $u \in \mathrm{Pref}(v)$ ou \linebreak $v \in \mathrm{Pref}(u)$.
  \end{question}

  On dit qu'un langage $L$ sur $\Sigma$ est \emph{sans préfixe}
  si aucun mot de $L$ n'est préfixe propre d'un autre mot de $L$. 
  \begin{question}
  \item Montrer que le produit de deux langages sans préfixe est sans préfixe. 
  \item Montrer, par récurrence sur $|u|$, que
    $$\forall a,b \in \Sigma,\ \forall u \in \Sigma^\star,\ u\cdot a = b\cdot u \ \Rightarrow\ a = b\ \land\ u \in \{a\}^\star.$$
  \end{question}

\end{exercice}

\endgroup
\endinput
