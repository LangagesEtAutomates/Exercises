% SPDX-License-Identifier: CC-BY-SA-4.0
% Part: Introduction
% Section: Mathematical and Logical recalls
% Exercise: Equipotence

\begingroup

\begin{exercice}[Equipotence]\label{exo:introduction/maths/equipotence}

  Soient $E$ un ensemble et $A, B \in \mathscr{P}(E)$ deux sous-ensembles de $E$. On dit que \emph{$A$ est équipotent à $B$}, noté
  $A \sim B$, s'il existe une bijection de $A$ dans $B$.
  \begin{question}
  \item Lister tous les éléments de $\mathscr{P}(\{a, b, c\})$.
  \item Si $E = \{a, b, c\}$, montrer que $\{a, b\} \sim \{b, c\}$. 
  \item Montrer que, pour tout ensemble $E$, $\sim$ est une relation d'équivalence sur  $\mathscr{P}(E)$. 
  \item Déterminer l'ensemble quotient de $\mathscr{P}(\{a, b, c\})$ par $\sim$. 
  \end{question}

\end{exercice}

\endgroup
\endinput
