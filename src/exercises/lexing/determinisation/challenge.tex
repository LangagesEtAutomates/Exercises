% SPDX-License-Identifier: CC-BY-SA-4.0
% Part: Lexical Analysis
% Section: Deterministic and complete automata
% Challenge exercise: Complexity of determinisation

\begingroup

\begin{exercice}[Mini-défi -- Complexité de la déterminisation]\label{exo:lexing/determinisation/challenge}

  Le but de cet exercice est de montrer que, pour certains langages,
  le nombre d'états d'un automate fini déterministe reconnaissant un langage
  peut être exponentiellement plus grand que le nombre d'états
  d'un automate fini non déterministe reconnaissant ce même langage.

  Soit $n\in \mathbb{N}$ un entier, et soit $\Sigma = \{a, b\}$.
  On considère le langage $L_n = \{u a v \mid u, v \in \Sigma^\star \text{ et } |v| = n \}$.

  \begin{question}
  \item Construire un automate fini non déterministe à $n+1$ états qui reconnaît le langage $L_n$.
  \item Appliquer la construction des sous-ensembles pour déterminiser l'automate précédent. 
  \item Montrer que cet automate déterministe est minimal.
  \item Quel est le nombre d'états de l'automate déterministe obtenu ? 
  \end{question}

\end{exercice}

\endgroup
\endinput

