% SPDX-License-Identifier: CC-BY-SA-4.0
% Part: Lexical Analysis
% Section: Deterministic and complete automata
% Exercise: Determinisation using Rabin-Scott's algorithm

\begingroup

\begin{exercice}[Propriétés des automates]\label{exo:lexing/determinisation/determinisation}

  Pour chacun des automates suivant définis sur l'alphabet $\{a,b\}$ : 
  \begin{itemize}
  \item Dire s'il est unitaire, standard, normalisé, $\varepsilon$-libre, déterministe et complet.
  \item Si l'automate n'est pas complet, le compléter. 
  \item Si l'automate n'est pas déterministe, donner l'automate déterministe équivalent par la méthode de construction des 
    sous-ensembles de Rabin et Scott.
  \end{itemize}

  \begin{question}

  \item 
    \begin{tikzpicture}[automaton, size=15mm, baseline=(0.base)]
      \state[initial]   (0) at (0,0) {$0$}; 
      \state            (1) at (1,0) {$1$}; 
      \state[accepting] (2) at (2,0) {$2$}; 
      \path (0) edge             node {$a$} (1);
      \path (1) edge             node {$b$} (2);
      \path (2) edge[loop above] node {$a$} (2);
    \end{tikzpicture}    

  \item \begin{tikzpicture}[automaton, size=15mm, baseline=(0.base)]
    \state[initial]   (0) at (0,0) {$0$}; 
    \state            (1) at (1,0) {$1$}; 
    \state[accepting] (2) at (2,0) {$2$}; 
    \path (0) edge[loop above] node {$a, b$}  (0);
    \path (1) edge[loop above] node {$a, b$}  (1);
    \path (2) edge[loop above] node {$a, b$}  (2);
    \path (0) edge             node {$a$}     (1);
    \path (1) edge             node {$a$}     (2);
  \end{tikzpicture}
    
  \item \begin{tikzpicture}[automaton, size=15mm, baseline=(1.base)]
    \state[initial]   (1) at (0,1) {$1$}; 
    \state            (2) at (1,2) {$2$}; 
    \state            (3) at (1,0) {$3$}; 
    \state[accepting] (4) at (2,1) {$4$}; 

    \path (1) edge[bend left] node {$a$} (2);
    \path (1) edge            node {$b$} (3);
    \path (2) edge            node {$a$} (1);
    \path (2) edge[bend left] node {$b$} (4);
    \path (3) edge[bend left] node {$b$} (1);
    \path (3) edge            node {$a$} (4);
    \path (4) edge            node {$b$} (2);
    \path (4) edge[bend left] node {$a$} (3);
  \end{tikzpicture}

          \item 
    \begin{tikzpicture}[automaton, size=15mm, baseline=(1.base)]
      \state[initial, accepting] (1) at (0,1) {$1$}; 
      \state                     (2) at (1,1) {$2$}; 
      \state                     (3) at (2,1) {$3$}; 
      \state                     (4) at (0,0) {$4$}; 
      \state                     (5) at (1,0) {$5$}; 
      \state[accepting]          (6) at (2,0) {$6$}; 

      \path (1) edge[loop above] node {$a$}   (1);
      \path (4) edge[loop below] node {$a,b$} (4);
      \path (5) edge[loop below] node {$b$}   (5);
      \path (1) edge             node {$b$}   (2);
      \path (1) edge             node {$a$}   (4);
      \path (2) edge             node {$a$}   (3);
      \path (2) edge             node {$b$}   (4);
      \path (3) edge[bend left]  node {$a$}   (6);
      \path (6) edge[bend left]  node {$a$}   (3);
      \path (6) edge             node {$a$}   (5);
    \end{tikzpicture}

  \item 
    \begin{tikzpicture}[automaton, size=15mm, baseline=(1.base)]
      \state[initial]   (1)  at (0,2) {$1$}; 
      \state            (2)  at (1,2) {$2$}; 
      \state            (3)  at (0,1) {$3$}; 
      \state            (4)  at (2,2) {$4$}; 
      \state            (5)  at (3,2) {$5$}; 
      \state            (6)  at (4,2) {$6$}; 
      \state            (7)  at (1,1) {$7$}; 
      \state            (8)  at (2,1) {$8$}; 
      \state[initial]   (9)  at (0,0) {$9$}; 
      \state            (10) at (1,0) {$10$}; 
      \state[accepting] (11) at (4,1) {$11$}; 

      \path (1)  edge            node[swap] {$\varepsilon$} (2);
      \path (2)  edge            node[swap] {$\varepsilon$} (4);
      \path (4)  edge            node       {$a$}           (5);
      \path (5)  edge            node[swap] {$\varepsilon$} (6);
      \path (6)  edge            node       {$\varepsilon$} (11);
      \path (1)  edge            node       {$\varepsilon$} (3);
      \path (3)  edge            node       {$a$}           (7);
      \path (7)  edge            node       {$b$}           (8);
      \path (8)  edge            node       {$\varepsilon$} (11);
      \path (9)  edge            node       {$\varepsilon$} (10);
      \path (10) edge            node[swap] {$b$}           (8);
      \path (2)  edge[bend left] node       {$\varepsilon$} (6);
      \path (5)  edge[bend left] node       {$\varepsilon$} (4);
    \end{tikzpicture}
    
  \end{question}

\end{exercice}

\endgroup
\endinput

