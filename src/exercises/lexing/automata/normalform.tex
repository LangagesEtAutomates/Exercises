% SPDX-License-Identifier: CC-BY-SA-4.0
% Part: Lexical Analysis
% Section: Finite state automata
% Exercise: Revision exercise

\begingroup

\begin{exercice}[Automates standards et normalisés]\label{exo:lexing/automata/normalform}

  \begin{question}
  \item Donner un automate fini standard  équivalent à l'automate ci-dessous défini sur l'alphabet $\{a,b,c\}$. 

  \begin{center}
    \begin{tikzpicture}[automaton, x=20mm, y=20mm]
      \state[initial]   (1) at (0,1) {$1$}; 
      \state            (2) at (1,1) {$2$}; 
      \state[accepting] (3) at (2,1) {$3$}; 
      \state[initial]   (4) at (0,0) {$4$}; 
      \state            (5) at (1,0) {$5$}; 
      \state[accepting] (6) at (2,0) {$6$}; 

      \path (1) edge             node {$a$} (2);
      \path (1) edge             node {$a$} (4);
      \path (2) edge             node {$b$} (5);
      \path (2) edge[bend left]  node {$c$} (3);
      \path (3) edge[bend left]  node {$a$} (2);
      \path (4) edge             node {$b$} (5);
      \path (5) edge             node {$c$} (3);
      \path (5) edge[bend left]  node {$c$} (6);
      \path (6) edge[bend left]  node {$b$} (5);
      \path (6) edge[loop right] node {$a$} (6);
    \end{tikzpicture}
  \end{center}

  \item Donner un automate fini normalisé équivalent. 
  \item Montrer qu'un automate fini normalisé n'est jamais complet.
  \end{question}

\end{exercice}

\endgroup
\endinput
