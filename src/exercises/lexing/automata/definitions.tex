% SPDX-License-Identifier: CC-BY-SA-4.0
% Part: Lexical Analysis
% Section: Finite state automata
% Exercise: Definitions on automata

\begingroup

\begin{exercice}[Configurations et suites d'actions d'un automate]\label{exo:lexing/automata/definitions}

  On se donne l'automate $A$ suivant :
  \begin{tikzpicture}[automaton, baseline=(1.base), x=17mm, y=17mm]
    \state[initial]   (1) at (0,1) {$1$}; 
    \state            (2) at (1,1) {$2$}; 
    \state[accepting] (3) at (2,1) {$3$}; 
    \state            (4) at (0,0) {$4$}; 
    \state            (5) at (1,0) {$5$}; 
    \state[accepting] (6) at (2,0) {$6$}; 

    \path (1) edge[loop above] node       {$a$} (1);
    \path (1) edge             node[swap] {$a$} (4);
    \path (1) edge             node       {$b$} (2);
    \path (2) edge             node       {$a$} (3);
    \path (2) edge[loop above] node       {$b$} (2);
    \path (2) edge             node       {$b$} (6);
    \path (3) edge[loop above] node       {$a$} (3);
    \path (4) edge             node[swap] {$b$} (5);
    \path (4) edge             node       {$b$} (2);
    \path (5) edge             node[swap] {$b$} (2);
  \end{tikzpicture}

  \begin{question}
  \item Donner la définition mathématique de l'automate $A$ sous forme d'un tuple. 
  \end{question}
  
  Donner le graphe de la relation d'action entre toutes les configurations accessibles
  dans la reconnaissance de chacun des mots suivants et en déduire s'ils sont reconnus :
  \begin{question}
  \item $ab$
  \item $\varepsilon$
  \item $aabbaa$.
  \item Quel est le langage reconnu par l'automate $A$ ?
  \end{question}

\end{exercice}

\endgroup
\endinput
