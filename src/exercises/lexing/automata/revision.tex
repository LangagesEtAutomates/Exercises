% SPDX-License-Identifier: CC-BY-SA-4.0
% Part: Lexical Analysis
% Section: Finite state automata
% Exercise: Revision exercise

\begingroup

\begin{exercice}[Révisions -- Automates finis]\label{exo:lexing/automata/revision}

  On considère l'automate fini $A = \left\langle \{a, b\}, \{0,1,2,3\}, \{0\}, \{2\}, \left\{\begin{array}{lll}
    \langle 0, a, 1\rangle, &
    \langle 1, b, 2\rangle, &
    \langle 2, b, 2\rangle, \\
    \langle 2, a, 3\rangle, &
    \langle 3, a, 3\rangle, &
    \langle 3, b, 3\rangle
  \end{array}\right\} \right\rangle$.

  \begin{question}

  \item Donner la représentation graphique et la représentation matricielle de $A$.

  \item Simuler l'exécution de $A$ sur les mots suivants et dire s'ils sont acceptés :
    $$ab,\quad abb,\quad aab,\quad \varepsilon.$$

  \item L'automate $A$ est-il déterministe ? Est-il complet ? 
    
  \item Décrire en français le langage $\mathcal{L}(A)$.

  \end{question}

  Construire un automate fini déterministe reconnaissant chacun des langages suivants :
  
  \begin{question}

  \item les mots de $\{a,b\}^\star$ se terminant par le facteur $aba$ ;

  \item les mots de $\{a,b\}^\star$ ne contenant pas le facteur $bb$ ;

  \end{question}

\end{exercice}

\endgroup
\endinput
