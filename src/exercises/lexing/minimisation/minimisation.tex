% SPDX-License-Identifier: CC-BY-SA-4.0
% Part: Lexical Analysis
% Section: Minimal automata
% Exercise: Minimisation using Moore's algorithm

\begingroup

\begin{exercice}[Automates minimaux et émondés]\label{exo:lexing/minimisation/minimisation}

  Pour les deux automates suivants définis sur l'alphabet $\{a,b, c\}$ : 
  \begin{itemize}
  \item Dire quels états sont accessibles, inaccessibles, co-accessibles, stériles, utiles et inutiles.
  \item Émonder les automates qui ne le sont pas déjà. 
  \item Donner l'automate minimal équivalent par la méthode de Moore.
  \end{itemize}

  \begin{question}
  
  \item 
    \begin{tikzpicture}[automaton, size=20mm, baseline=(1.base)]
      \state[initial]   (1) at (0,1) {$1$}; 
      \state[accepting] (2) at (1,1) {$2$}; 
      \state            (3) at (2,1) {$3$}; 
      \state            (4) at (3,1) {$4$}; 
      \state            (5) at (0,0) {$5$}; 
      \state[accepting] (6) at (1,0) {$6$}; 
      \state            (7) at (2,0) {$7$}; 
      \state            (8) at (3,0) {$8$}; 

      \path (2) edge[loop above] node       {$a$}      (2);
      \path (4) edge[loop above] node       {$a$}      (4);
      \path (6) edge[loop below] node       {$b$}      (6);
      \path (1) edge             node       {$a$}      (2);
      \path (2) edge             node       {$c$}      (3);
      \path (3) edge             node       {$b$}      (4);
      \path (1) edge             node[swap] {$b, c$}   (6);
      \path (2) edge[bend left]  node       {$b$}      (6);
      \path (6) edge[bend left]  node       {$a$}      (2);
      \path (4) edge             node       {$a, c$}   (8);
      \path (7) edge             node       {$a$}      (8);
      \path (8) edge             node       {$a$}      (3);
      \path (5) edge             node[swap] {$a, b, c$} (6);
    \end{tikzpicture}

  \item 
      \begin{tikzpicture}[automaton, size=20mm, baseline=(0.base)]
        \state[initial]   (0) at (0,1) {$0$}; 
        \state            (1) at (1,1) {$1$}; 
        \state            (2) at (2,1) {$2$}; 
        \state            (3) at (0,0) {$3$}; 
        \state            (4) at (1,0) {$4$}; 
        \state[accepting] (5) at (2,0) {$5$}; 

        \path (0) edge[loop above] node       {$c$}       (0);
        \path (1) edge[loop above] node       {$c$}       (1);
        \path (2) edge[loop above] node       {$b$}       (2);
        \path (3) edge[loop below] node       {$c$}       (3);
        \path (4) edge[loop below] node       {$b$}       (4);
        \path (5) edge[loop right] node       {$a, b, c$} (5);
        \path (0) edge             node       {$a$}       (1);
        \path (0) edge             node[swap] {$b$}       (3);
        \path (2) edge             node       {$c$}       (5);
        \path (4) edge             node       {$c$}       (5);
        \path (1) edge[bend left]  node       {$a$}       (2);
        \path (2) edge[bend left]  node       {$a$}       (1);
        \path (1) edge             node       {$b$}       (3);
        \path (3) edge[bend left]  node       {$b$}       (1);
        \path (3) edge             node       {$a$}       (4);
        \path (4) edge[bend left]  node       {$a$}       (3);
      \end{tikzpicture}
  
  \end{question}

\end{exercice}

\endgroup
\endinput
