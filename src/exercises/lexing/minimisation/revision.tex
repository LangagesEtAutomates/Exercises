% SPDX-License-Identifier: CC-BY-SA-4.0
% Part: Lexical Analysis
% Section: Minimal automata
% Exercise: Revisions exercise: Equivalence between automata

\begingroup

\begin{exercice}[Révisions -- \'Equivalence entre automates] \label{exo:lexing/minimisation/revision}

  \begin{question}

  \item Donner l'automate minimal équivalent à l'automate déterministe suivant par la méthode de Moore.
    \begin{center}
      \begin{tikzpicture}[automaton, size=20mm]
          \state[initial]   (0) at (0,1) {$0$}; 
          \state[accepting] (1) at (1,1) {$1$}; 
          \state            (2) at (1,0) {$2$}; 
          \state[accepting] (3) at (2,0) {$3$}; 
          \state[accepting] (4) at (2,1) {$4$}; 
          \state            (5) at (0,0) {$5$}; 

          \path (3) edge[loop right] node       {$b$}    (3);
          \path (4) edge[loop right] node       {$b$}    (4);
          \path (5) edge[loop left]  node       {$a, b$} (5);
          \path (0) edge             node       {$a$}    (1);
          \path (0) edge             node       {$b$}    (5);
          \path (1) edge             node       {$a$}    (2);
          \path (1) edge             node       {$b$}    (4);
          \path (2) edge             node       {$b$}    (5);
          \path (4) edge             node[swap] {$a$}    (2);
          \path (3) edge[bend left]  node       {$a$}    (2);
          \path (2) edge[bend left]  node       {$a$}    (3);
      \end{tikzpicture}
    \end{center}

  \item Minimiser l'automate suivant.
    \begin{center}
    \begin{tikzpicture}[automaton, size=20mm, baseline=(1.base)]
      \state[initial, accepting] (1) at (0,1) {$1$}; 
      \state                     (2) at (1,1) {$2$}; 
      \state                     (3) at (2,1) {$3$}; 
      \state                     (4) at (0,0) {$4$}; 
      \state                     (5) at (1,0) {$5$}; 
      \state[accepting]          (6) at (2,0) {$6$}; 

      \path (1) edge[loop above] node {$a$}   (1);
      \path (4) edge[loop below] node {$a,b$} (4);
      \path (5) edge[loop below] node {$b$}   (5);
      \path (1) edge             node {$b$}   (2);
      \path (1) edge             node {$a$}   (4);
      \path (2) edge             node {$a$}   (3);
      \path (2) edge             node {$b$}   (4);
      \path (3) edge[bend left]  node {$a$}   (6);
      \path (6) edge[bend left]  node {$a$}   (3);
      \path (6) edge             node {$a$}   (5);
    \end{tikzpicture}
    \end{center}

    \item Montrer qu'un automate minimal a au plus un état stérile.
    
  \end{question}
  
\end{exercice}

\endgroup
\endinput
