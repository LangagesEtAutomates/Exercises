% SPDX-License-Identifier: CC-BY-SA-4.0
% Part: Lexical Analysis
% Section: Expressiveness
% Exercise: Complement of a rational language

\begingroup

\begin{exercice}[Mini-défi -- Complément d'un langage rationnel] \label{exo:lexing/minimisation/inverse}

  Le but de cet exercice est de trouver une expression rationnelle décrivant le langage $L$ des mots sur l'alphabet $\Sigma=\{a,b,c\}$ ne contenant pas de facteur $abc$. 
  \begin{question}
  \item Proposez une expression rationnelle décrivant le langage $\overline{L}$ des mots contenant le facteur $abc$.
  \item Donnez l'automate fini non déterministe obtenu en appliquant l'algorithme de Thompson sur l'expression rationnelle de la question précédente.
  \item Donnez un automate fini déterministe et complet reconnaissant le langage $\overline{L}$.
  \item Donnez un automate fini déterministe et complet reconnaissant le langage $L$.
  \item Minimisez l'automate obtenu dans la question précédente.
  \item Donnez une expression rationnelle décrivant le langage $L$.
  \item En déduire une méthode générale permettant, à partir d'une expression rationnelle décrivant un langage $L$,
      de construire une expression rationnelle décrivant son complémentaire $\overline{L}$.
  \end{question}

\end{exercice}

\endgroup
\endinput
