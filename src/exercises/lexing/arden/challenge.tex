% SPDX-License-Identifier: CC-BY-SA-4.0
% Part: Lexical Analysis
% Section: Arden's lemma and identities
% Challenge exercise: Extending Arden's lemma

\begingroup

\begin{exercice}[Mini-défi -- Étendre le lemme d'Arden]\label{exo:lexing/arden/challenge}

  Soient $\Sigma$ un alphabet et $A, B \subseteq \Sigma^\star$ deux langages. 
  On considère l'équation en langages $X = A\cdot X \mid B$.
  D'après le lemme d'Arden, nous savons que l'équation $X = A\cdot X\mid B$ a une seule
  solution quand $\varepsilon\notin A$.
  On suppose $\varepsilon \in A$. Le but de l'exercice est de prouver le théorème suivant :
  
  $$
  \text{Un langage }X\text{ est solution de }X = A\cdot X \mid B 
  \iff \exists L \subseteq \Sigma^\star,\; X = A^\star \cdot (L \mid B).
  $$

  \begin{question}
  \item Montrer que $\Sigma^\star$ est solution de $X = A\cdot X \mid B$.
  \item Montrer que, pour tout $L \subseteq \Sigma^\star$, le langage $A^\star \cdot (L \mid B)$ est solution de $X = A\cdot X \mid B$.
  \item Soit $M$ une solution de $X = A\cdot X \mid B$. Montrer que $M \subseteq A^\star \cdot (M \mid B)$.
  \item Montrer, par récurrence sur $n$, que $\forall n\in\mathbb{N},\; A^n \cdot (M \mid B) \subseteq M$.
  \item En déduire $M = A^\star \cdot (M \mid B)$.
  \item Conclure que $X$ est solution de $X = A\cdot X \mid B$ si et seulement si 
    $\exists L \subseteq \Sigma^\star$ tel que $X = A^\star \cdot (L \mid B)$.
  \end{question}

\end{exercice}

\endgroup
\endinput
