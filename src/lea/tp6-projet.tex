\ifdefined\CORRECTION
\documentclass[correction]{td}
\else
\documentclass{td}
\fi

\inputexercisepath[src/corrections/TP]{src/exercises}

\usepackage[T1]{fontenc}
\usepackage[hidelinks]{hyperref}
\usepackage{amsmath,amssymb, mathrsfs, stmaryrd}
\newcommand\eqdef{\stackrel{\text{def}}{=}}

\usepackage{algos}
\usepackage{statemachines}
\usepackage{trees}

\codeUE{XLG4IU020}
\intituleUE{Langages et Automates}
\cursus{L2 Informatique}

\author{Matthieu \textsc{Perrin}}

\logo{src/img/logoUN.png}
\institution{Nantes Université}

\typeTP[6]
\title{Projet --- Enregistrements}

\hypersetup{
  pdftitle  = {Langages et Automates - Projet - Enregistrements},
  pdfauthor = {Matthieu Perrin},
  pdfsubject  = {Projet de L2 en Langages et Automates},
  pdfkeywords = {compilateur, lexer, parser, AST, typage, enregistrements}
}

\begin{document}

\maketitle

\begin{exercice}[Enregistrements]

  Le but de ce projet est d'étendre le langage avec des \emph{enregistrements} (records).
  Le code de départ fournit déjà le lexer, le parseur, l'AST, un interpréteur,
  et une analyse statique de types.

  \begin{question}

  \item \textbf{Déclarations d'enregistrements.}
    Ajoutez une section \lstinline|enregistrements| au langage, contenant des définitions de la forme :
    \begin{lstlisting}
      enregistrements
        enregistrement Point début
          x : entier;
          y : entier;
        fin
        enregistrement Personne début
          nom : chaine;
          age : entier;
        fin
    \end{lstlisting}
    À faire :
    \begin{itemize}
      \item ajouter les mots-clés au lexer JFlex ;
      \item étendre le parseur CUP (ajout de règles de grammaire) ;
      \item étendre l'AST avec un type \lstinline|TRecord(name)| et une structure de données
        associant à chaque nom d'enregistrement la table de ses champs \lstinline|Map<Identifier,Type>| ;
      \item construire une \emph{table des types} (table de symboles) permettant de retrouver la
        déclaration d'un enregistrement à partir de son nom.
    \end{itemize}

  \item \textbf{Valeurs enregistrement.}
    Ajoutez des valeurs d'enregistrement et une syntaxe de construction :
    \begin{lstlisting}
      p <- nouveau Point(x = 1, y = 2);
    \end{lstlisting}
    À faire :
    \begin{itemize}
      \item ajouter les lexèmes nécessaires (\lstinline|nouveau|, \lstinline|.|, \lstinline|=|, \lstinline|,|) ;
      \item étendre l'AST avec une valeur record (association \lstinline|Map<Identifier,Value>|) ;
      \item étendre le parseur pour reconnaître la construction \lstinline|nouveau Nom(... )| ;
      \item implémenter l'évaluation dans l'interpréteur.
    \end{itemize}

  \item \textbf{Accès aux champs.}
    Ajoutez l'opérateur point (associatif à gauche) :
    \begin{lstlisting}
      écrire(p.x + p.y);
    \end{lstlisting}
    À faire :
    \begin{itemize}
      \item ajouter le lexème \lstinline|.| au lexer ;
      \item étendre la grammaire CUP en respectant la priorité (l'accès champ doit être plus prioritaire
        que \lstinline|*|, \lstinline|+|, etc.) ;
      \item étendre l'AST avec un nœud \lstinline|FieldAccess(expr, field)| ;
      \item implémenter l'évaluation dans l'interpréteur (erreur si champ absent).
    \end{itemize}

  \item \textbf{Analyse statique de types.}
    Adaptez l'analyse statique de types afin de traiter :
    \begin{itemize}
      \item les types enregistrement (déclaration de variables \lstinline|p : Point;|) ;
      \item la construction \lstinline|nouveau Point(... )| :
        \begin{itemize}
          \item l'enregistrement doit exister ;
          \item tous les champs doivent être initialisés \emph{exactement une fois} ;
          \item le type de chaque champ doit être respecté ;
        \end{itemize}
      \item l'accès \lstinline|p.x| : \lstinline|p| doit avoir un type enregistrement et \lstinline|x| doit être un champ existant.
    \end{itemize}

  \item \textbf{Programmes de validation.}
    Testez votre implémentation sur les cas suivants (au minimum) :
    \begin{itemize}
      \item création + accès champ valides ;
      \item accès à un champ inexistant ;
      \item construction avec champ manquant ;
      \item construction avec champ en trop ;
      \item construction avec champ de mauvais type.
    \end{itemize}

  \end{question}

\end{exercice}

\end{document}
