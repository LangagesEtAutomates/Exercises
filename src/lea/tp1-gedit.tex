\ifdefined\CORRECTION
\documentclass[correction]{td}
\else
\documentclass{td}
\fi

\inputexercisepath[src/corrections/TP]{src/exercises}

\usepackage[T1]{fontenc}
\usepackage[hidelinks]{hyperref}
\usepackage{amsmath,amssymb, mathrsfs, stmaryrd}
\newcommand\eqdef{\stackrel{\text{def}}{=}}

\usepackage{algos}
\usepackage{statemachines}
\usepackage{trees}

\codeUE{XLG4IU020}
\intituleUE{Langages et Automates}
\cursus{L2 Informatique}

\author{Matthieu \textsc{Perrin}}

\logo{src/img/logoUN.png}
\institution{Nantes Université}

\typeTP[1]
\title{Coloration syntaxique dans Gedit}

\hypersetup{
  pdftitle  = {Langages et Automates - TP 1 - Coloration syntaxique dans Gedit},
  pdfauthor = {Matthieu Perrin},
  pdfsubject  = {TP de L2 en Langages et Automates},
  pdfkeywords = {coloration syntaxique, expressions rationnelles, analyse lexicale, Gedit}
}

\begin{document}

\maketitle

Le but de ce TP est de configurer l'éditeur de texte Gedit pour 
la coloration syntaxique d'un langage algorithmique appelé \texttt{Algo}.
Gedit utilise la bibliothèque \texttt{gtksourceview}
qui se base sur des expressions rationnelles pour reconnaître des catégories lexicales
(mots-clés, identifiants, nombres, commentaires, etc.).
Ces règles de reconnaissance du langage sont décrites dans un fichier \texttt{.lang} au format XML,
et les couleurs dans un fichier de thème \texttt{.xml}.
Ce TP est indépendant du projet que nous développerons dans les séances suivantes.

\begin{exercice}[Mise en place]

  \begin{question}
  \item Clonez le dépôt du TP :
    \begin{lstlisting}[gobble=4]
      git clone https://github.com/LangagesEtAutomates/lea-tp1-gedit
      cd lea-tp1-gedit
    \end{lstlisting}

  \item Créez les dossiers de configuration de \texttt{gtksourceview} :
    \begin{lstlisting}[gobble=4]
      mkdir -p ~/.local/share/gtksourceview-4/styles
      mkdir -p ~/.local/share/gtksourceview-4/language-specs
    \end{lstlisting}

  \item Créez des \emph{liens symboliques} depuis votre dépôt vers les répertoires de Gedit :
    \begin{lstlisting}[gobble=4]
      ln -s "$(pwd)/src/TPLea.xml" ~/.local/share/gtksourceview-4/styles/TPLea.xml
      ln -s "$(pwd)/src/algo.lang" ~/.local/share/gtksourceview-4/language-specs/algo.lang
    \end{lstlisting}

    \emph{Remarque :} si un lien existe déjà, vous pouvez le supprimer avec \lstinline|rm| puis le recréer.

  \item Ouvrez le fichier d'exemple avec \lstinline{gedit} :
    \begin{lstlisting}[gobble=4]
      gedit test/programme.algo
    \end{lstlisting}

  \item Dans le menu \og Préférences $\rightarrow$ Police et couleurs \fg, sélectionnez le thème \lstinline{TPLea}.

  \item Dans le menu \og Affichage $\rightarrow$ Mode de coloration \fg, sélectionnez le langage \lstinline{Algo}.
  \end{question}

  Vous devriez avoir un affichage en mode sombre, avec certains mots-clés écrits en bleu clair.
\end{exercice}



\paragraph{Syntaxe des expressions rationnelles}
Les expressions rationnelles sont des motifs permettant de décrire un ensemble de chaînes de caractères.
Elles sont largement utilisées pour effectuer des recherches et des manipulations avancées de texte dans divers outils et langages de programmation.
Elles combinent des caractères ordinaires (lettres, chiffres, symboles) et des \emph{métacaractères} ayant une signification spéciale.
Nous présentons ici la syntaxe utilisée dans des outils tels que \texttt{egrep} et \texttt{Gedit}.
Dans ce TP, nous utiliserons indifféremment les termes \emph{expression rationnelle} et \emph{expression régulière (regex)}.

\begin{description}
\item[Correspondance simple.] La plupart des caractères correspondent à eux-mêmes :
  \begin{itemize}
  \item \lstinline{a} correspond au caractère \texttt{a}.
  \item \lstinline{bonjour} correspond exactement à la chaîne \og bonjour \fg.
  \end{itemize}

\item[Classes de caractères et intervalles.] Une classe de caractères permet de rechercher un caractère parmi un ensemble défini.
  Un intervalle peut être défini avec un tiret (\lstinline{-}) entre deux bornes. 
  Le complément d'une classe est noté avec un accent circonflexe \lstinline{^}\footnote{Clavier azerty : appuyer sur \texttt{Alt Gr} + \texttt{9}.} en début de liste.
  \begin{itemize}
  \item \lstinline{[aeiou]} correspond à une voyelle minuscule.
  \item \lstinline{[a-d]} est équivalent à \lstinline{[abcd]}.
  \item \lstinline{[0-9]} correspond à un chiffre.
  \item \lstinline{[A-Z]} correspond à une lettre majuscule.
  \item \lstinline{[a-zA-Z]} correspond à toute lettre alphabétique.
  \item \lstinline{[^0-9]} correspond à tout caractère qui \textbf{n'est pas} un chiffre.
  \end{itemize}

\item[Classes prédéfinies.] Certains raccourcis permettent de représenter des classes courantes :
  \begin{itemize}
  \item \lstinline{.} correspond à n'importe quel caractère, sauf le passage à la ligne (\lstinline{[^\n]}).
  \item \lstinline{\d} correspond aux chiffres (\lstinline{[0-9]}).
  \item \lstinline{\w} correspond aux caractères alphanumériques et l'underscore (\lstinline{[a-zA-Z0-9\_]}).
  \item \lstinline{\s} correspond aux espaces (\lstinline{[ \t\n\r]}).
  \item \lstinline{\D}, \lstinline{\W} et \lstinline{\S} sont les complémentaires respectifs de \lstinline{\d}, \lstinline{\w} et \lstinline{\s}.
  \end{itemize}

\item[Caractères spéciaux et échappement.]
  Les métacaractères ont une signification spéciale et doivent être échappés avec un antislash \lstinline{\}\footnote{Clavier azerty : appuyer sur \texttt{Alt Gr} + \texttt{8}.} pour être interprétés littéralement.
    \begin{itemize}
    \item \lstinline{\.} permet de rechercher un point.
    \item \lstinline{\(} permet de rechercher une parenthèse ouvrante.
    \end{itemize}

  \item[Métacaractères de position.] Ces symboles définissent où l'expression doit être située dans la chaîne :
    \begin{itemize}
    \item \lstinline{^} (accent circonflexe) : début de ligne (\lstinline{^Bonjour} trouve les \og Bonjour \fg{} en début de ligne).
    \item \lstinline{$} (dollar) : fin de ligne (\lstinline{monde$} trouve \og monde \fg{} uniquement en fin de ligne).
    \item \lstinline{\b} : limite de mot (\lstinline{\bchat\b} trouve exactement \og chat \fg, mais pas \og chatière \fg).
    \end{itemize}

  \item[Quantificateurs de répétition.] Les quantificateurs définissent combien de fois un motif peut apparaître :
    \begin{itemize}
    \item \lstinline{?} : 0 ou 1 occurrence (\lstinline{colou?r} trouve \og color \fg{} et \og colour \fg).
    \item \lstinline{*} : 0 ou plusieurs occurrences (\lstinline{ab*c} trouve \og ac \fg, \og abc \fg, \og abbc \fg, etc.).
    \item \lstinline{+} : 1 ou plusieurs occurrences (\lstinline{ab+c} trouve \og abc \fg, \og abbc \fg, mais pas \og ac \fg).
    \item \lstinline|{n}| : exactement \lstinline{n} occurrences (\lstinline|a{3}| trouve \og aaa \fg).
    \item \lstinline|{n,}| : au moins \lstinline{n} occurrences (\lstinline|a{2,}| trouve \og aa \fg, \og aaa \fg, etc.).
    \item \lstinline|{n,m}| : entre \lstinline{n} et \lstinline{m} occurrences (\lstinline|a{2,4}| trouve \og aa \fg, \og aaa \fg{} ou \og aaaa \fg).
    \end{itemize}

  \item[Opérateurs de combinaison.] Des opérateurs permettent de structurer des motifs plus complexes :
    \begin{itemize}
    \item Juxtaposer deux expressions permet de les concaténer (\lstinline{hello} recherche exactement \og hello \fg).
    \item L'opérateur \lstinline{|} signifie \emph{ou} (\lstinline{chat|chien} trouve \og chat \fg{} et \og chien \fg).
    \item Les parenthèses permettent de regrouper des sous-expressions (\lstinline{(ab|cd)+} recherche \og ab \fg, \og cd \fg, \og abcd \fg, etc.).
    \end{itemize}
\end{description}


\begin{exercice}[Recherche par expressions rationnelles dans Gedit]

  Ouvrez le fichier \lstinline{test/programme.algo} dans Gedit.
  Activez l'outil de recherche avec \lstinline{Ctrl+F}.
  Cliquez sur l'icône loupe et cochez l'option \og Expressions régulières \fg.

  \begin{question}

  \item Recherchez les motifs ``\lstinline{si}'' puis ``\lstinline{\bsi\b}''.
    Expliquez la différence observée.

  \item Un commentaire de ligne commence par \lstinline{//} et se termine en fin de ligne.
    Proposez une expression rationnelle qui reconnaît un commentaire de ligne. Vérifiez-la dans Gedit.

  \item Un nombre entier est représenté par une suite de chiffres ne commençant pas par $0$, sauf si le nombre est exactement $0$. 
    Proposez une expression rationnelle qui reconnaît les nombres. 
    Vérifiez votre proposition en l'utilisant dans la recherche de Gedit.

  \end{question}
\end{exercice}
\begin{correction}
  \begin{question}
  \item \lstinline{\b} correspond à une chaîne vide à l'extrémité d'un mot. En l'ajoutant en début et en fin de l'expression rationnelle,
    on reconnaît uniquement les occurrences du mot complet.
  \item Une réponse possible est \lstinline{//.*$}.
  \item Une réponse possible est \lstinline{\b([1-9][0-9]*|0)\b}.
  \end{question}
\end{correction}

\begin{exercice}[Coloration syntaxique]

  Ouvrez le fichier \lstinline{src/algo.lang} (avec l'éditeur de texte de votre choix).
  Le fichier suit une structure hiérarchique en XML et est utilisé pour définir les règles de reconnaissance des éléments syntaxiques du langage \texttt{Algo}.
  Chaque règle de coloration décrit un motif permettant de reconnaître une \emph{catégorie lexicale} (mots-clés, identifiants, nombres, commentaires, etc.).
  Chaque catégorie est associée à un \emph{contexte} auquel Gedit applique un \emph{style}.
  La structure générale du fichier est décrite ci-dessous.
  
  \begin{description}
  \item[Le contexte principal :]
    Le contexte \texttt{algo} est le point d'entrée. Il inclut tous les autres contextes nécessaires pour analyser le code.
    Il est activé quand Gedit ouvre un fichier en mode Algo.
  \item[Les contextes spécifiques :]
    Chaque élément syntaxique (mots-clés, nombres, commentaires, etc.) est défini dans un contexte séparé.
    Un \emph{style} (\texttt{style-ref}) est associé à chaque contexte pour spécifier comment les éléments seront colorés.
    Ces styles sont définis dans un fichier de style, ici le fichier \lstinline{TPLea.xml}.
  \item[Les mots-clés :]
    Les mots-clés sont définis avec des balises \texttt{<keyword>} dans un contexte. Par exemple : 
    \begin{lstlisting}
      <context id="keyword" style-ref="def:keyword">
        <keyword>algorithme</keyword>
        <keyword>variables</keyword>
      </context>
    \end{lstlisting}
  \item[Les expressions rationnelles :]
    Les règles de reconnaissance sont écrites en expressions rationnelles dans une balise \texttt{<match>}.
    Par exemple, pour reconnaître les identifiants (noms de variables ou des fonctions)
    selon l'expression rationnelle ``\lstinline{\b[A-Za-z_][A-Za-z_0-9]*\b}'',
    en appliquant le style \lstinline{def:identifier} pour les colorer dans l'éditeur (en blanc cassé italique dans le style \lstinline{TPLea}), on utilise :
    \begin{lstlisting}
      <context id="identifier" style-ref="def:identifier">
        <match extended="true">
          \b[A-Za-z_][A-Za-z_0-9]*\b
        </match>
      </context>
    \end{lstlisting}
  \end{description}

  \begin{question}
  \item Ajoutez une référence au contexte \lstinline{identifier} dans le contexte principal ``algo''.
    Dans un premier temps, ajoutez la ligne \lstinline{<context ref="identifier"/>}
    juste après la ligne \lstinline{<include>}. Relancez gedit et observez le résultat.
    Descendez la ligne \lstinline{<context ref="identifier"/>} juste avant la ligne \lstinline{</include>}.
    Comparez et expliquez la différence. 

  \item Modifiez la liste de mots-clés pour reconnaître les mots-clés manquants :
    ``\lstinline{et}'', ``\lstinline{ou}'' et ``\lstinline{fin}''. 

  \item En vous inspirant du context ``\texttt{identifier}'', ajoutez un contexte ``\texttt{number}'' pour reconnaître
    les nombres entiers tels que spécifiés dans l'exercice précédent, et appliquer le style \lstinline{def:number}.

  \item Ajoutez un contexte ``\texttt{char}'' pour reconnaître les caractères (entre guillemets simples).
    Ils devront appliquer le style \lstinline{def:character}. 

  \end{question}

\end{exercice}
\begin{correction}
  Voir le fichier \lstinline{src/algo.lang.correction} complété sur Madoc.
\end{correction}

\end{document}
